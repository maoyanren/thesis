\chapter{Conclusion}\label{chpt:conclusion}
In the current work, electroosmotic flow through a single nanopore under an applied electric field is studied both theoretically and numerically. Different scenarios have been considered, including zero/finite membrane thickness and zero/finite 
membrane surface charge density. The flow rate is computed using the reciprocal theorem as well as using full numerical simulations of the primitive equations. Eddies form in the vicinity of the nanopore due to ICEO even when the membrane surface is uncharged. The shape of the eddies are predicted by numerical simulations.

If the membrane surfaces have intrinsic surface charges, under the assumption of weak fields the net electric force acting on the fluid is $\rho_0\nabla\chi$, where $\rho_0$ is the local charge density at equilibrium (induced by the intrinsic surface charge on the membrane), and $\chi$ is the Ohmic potential. Physically this means that the driving force for the fluid flow may be obtained as a first approximation by assuming that the Ohmic electric field acts on the unperturbed equilibrium double layer charge and that the charge cloud is not disturbed by either the external electric field or the induced electroosmostic flow. By utilizing the reciprocal theorem, we show that the volumetric flow rate can be expressed as an integral over the membrane surface  without detailed knowledge of the flow field. For a circular hole in a zero-thickness membrane, the building blocks in the form of exact solutions for the Ohmic potential and pressure driven flow through a hole
are available theoretically. The integral is thus expressed in closed form and computed by quadrature. The result agrees with the full numerical simulation of the PNP-Stokes equations. The ratio of applied voltage to volumetric flow rate is defined as the  ``electroosmotic access resistance'' in analogy to the corresponding concept in the Ohmic conduction problem.
%Mao: I reversed this definition since "infinite Resistance" should correspond to Q=0.

When the nanopore is circular but has non-zero thickness (cylindrical), we patch the solution of electroosmosis through a zero-thickness pore with the classic theory of electroosmosis in a cylinder. The patched solution gives a total electroosmotic conductance as a weighed average of the electroosmotic conductance of a circular hole, and the electroosmotic conductance of a cylinder. The solution agrees with numerical simulation except at the thick Debye length limit $a\kappa \gg 1$. We show that the reason is that the charge overspill from the pore to the outside (and conversely the ``under spill'' from the EDL on the membrane into the pore) are significant when the Debye layer is thick. Under the circumstances the electroosmotic conductances for both a hole and a cylinder need to be corrected to account for this effect. We  provide such a correction to  the theory, and indeed, the modified solution does agree with the full numerical simulation.

When the membrane surfaces do not have surface charges, the problem has symmetry and as such no net flow is generated through the nanopore. This does not however mean that the fluid is stationary. Instead, ICEO effects can generate toroidal eddies within the nanopore. We provide theoretical analysis for this problem in the linear regime. Far away from the edge of the hole, the induced zeta potential is worked out based on the assumption that the radial variations can be neglected. Using the solution of the Laplace equation to calculate the tangential electric field, the Smoluchowski velocity is subsequently obtained. This is used as the outer solution, and its inner limit (towards the edge of the hole) is matched with a local solution of the ICEO problem. The local solution is worked out in 3 steps. First, the method of conformal mapping is used to obtain the Ohmic potential outside the Debye layer and within the solid wedge. Then we follow Thamida and Chang \cite{Thamida2002} and work out the induced zeta potential. Finally the Smoluchowski velocity is presented. We also use full numerical simulations to show how the shape of the toroidal eddies depend on the geometry of the pore. The interesting topology of vortex pairs form within the nanopore, and one particularly interesting case is when the eddies meet at one single stagnation point, dividing the fluid domain into eight separate regions.

Finally, the PNP-Stokes solvers are introduced. These solvers are based on the finite volume method using the OpenFOAM CFD library. The equations are solved iteratively due to the multiphysics coupling. When the polarizability of the solid membrane can be neglected, the jump condition of electric field at the solid-fluid membrane reduces to a Neumann boundary condition, and the computation can be restricted to the fluid region alone. When the polarizability of the solid membrane needs to be taken into account, the potential in the solid region needs to be computed as well, and specific solver and boundary condition are developed to account for this additional multi-region coupling.