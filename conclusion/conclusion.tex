\chapter{Conclusion}
In the current work, electroosmotic flow in a single nanopore under an applied electric field is studied both theoretically and numerically. Different scenarios have been considered, including the case when pore thickness is 0 or finite, and the case when the membrane surface charge density is finite or 0. A net flow exists when surface charge density is nonzero, and the flow rate is computed both from the reciprocal theorem and numerical simulations. Eddies form with the nanopore due to ICEO when membrane surface is uncharged. The shape of the eddies are predicted by numerical simulations.

If the membrane surfaces have intrinsic surface charges, under the assumption of weak field in the Debye-H\"{}uckel limit, asymptotic analysis is carried out and it is shown that the net electric force acting on the fluid is $\rho_0\nabla\chi$, where $\rho_0$ is the local charge density at equilibrium (induced by intrinsic surface charge on the membrane), and $\chi$ is the Ohmic potential. Physically this result means the driving force of the fluid flow is the Ohmic potential field acting on the equilibrium charge cloud, and that the charge cloud is not disturbed by either the external electric field or the induced electroosmostic flow. By utilizing the reciprocal theorem, we show the volumetric flow rate can be expressed as an integral, without solving the detailed flow field. For a circular hole on a zero-thickness membrane, the building blocks of the integral are available theoretically. The integral is then expressed in closed form and computed up to quadrature. The result agrees with full numerical simulation of the PNP-Stokes equations. The ratio of volumetric flow rate to the applied voltage is defined as ``electroosmotic access resistance'' in analogy to its counterpart in electrostatics.

When the nanopore is circular but has non-zero thickness (cylindrical), we patch the solution of electroosmosis through a zero-thickness pore with the classic theory of electroosmosis in a cylinder. The patched solution gives a total electroosmotic conductance as a weighed average of the electroosmotic conductance of a circular hole, and the electroosmotic conductance of a cylinder. The solution agrees with numerical simulation except at the thick Debye length limit $a\kappa \gg 1$. We show that the reason is that the charge overspill from the pore to outside (and vice versa) are significant when Debye layer is thick. Under this circumstance the electroosmotic conductances for both a hole and a cylinder need to be corrected. We theoretically provide such corrections, and the modified solution agrees with numerical simulation.

When the membrane surfaces do not have surface charges, the problem has symmetry such that no net flow will be generated through the nanopore. This does not mean no flow at all. Instead ICEO effects can generate toroidal eddies within the nanopore. We provide theoretical analysis for this problem in the linear regime. Far away from the edge of the hole, the induced zeta potential is worked out based on the assumption that the radial variations can be neglected. Using the solution of the Laplace equation to calculate the tangential electric field, the Smoluchowski velocity is subsequently obtained. This is used as the outer solution, and its inner limit (towards the edge of the hole) is matched with a local solution of the iceo problem. The local solution is worked out in 3 steps. First, the method of conformal mapping is used to obtain the Ohmic potential outside the Debye layer and within the solid wedge. Then we follow Thamida and Chang \cite{Thamida2002} and work out the induced zeta potential. Finally the Smoluchowski velocity is presented. We also use full numerical simulations to show that the shape of the toroidal eddies on the geometry of the pore. The interesting topology of vortex pairs form within the nanopore, and one particular interesting case is when the eddies meet at one single stagnation point, dividing the fluid domain into 8 separate regions.

Finally, the PNP-Stokes solvers are introduced. These solvers are based on finite volume method using the OpenFOAM CFD library. The equations are solved iteratively due to the multiphysics coupling. When the polarizability of the solid membrane can be neglected, the jump condition of electric field at the solid-fluid membrane reduces to a Neumann boundary condition, and the computation can be restricted to the fluid region alone. When the polarizability of the solid membrane needs to be taken into account, the potential in the solid region needs to be computed as well, and specific solver and boundary condition are developed to account for this additional multi-region coupling.